\documentclass{beamer}
\author{Peter Smaal}

\title{Learing \LaTeX}
\subtitle{What have we done}

\usepackage{graphicx}
\usepackage{fancyhdr}
\usepackage{apacite}

\pagestyle{fancy}
\rhead{Peter Smaal - 0908946}
\lhead{Eindopdracht \LaTeX}


\begin{document}
\newcommand*{\myfont}{\fontfamily{pcr}\selectfont}

\frame{\titlepage}


\begin{frame}
\frametitle{Using LaTeX}
LaTeX wordt gebruikt als tekstverwerker om documenten en presentaties etc. te creëren

\end{frame}

\begin{frame}
\frametitle{Welke editor gebruik ik?}
\myfont{Ik gebruik de editor "Texmaker". (Ik gebruik hier ook een andere font )}
\end{frame}

\begin{frame}
\begin{figure}
\frametitle{Een plaatje invoegen hebben we ook geleerd}
\includegraphics[width = 50mm]{Hogeschool-Rotterdam-Logo.pdf}
\caption{HR LOGO}
\label{fig: Hogeschool Rotterdam Logo}
\end{figure}
\end{frame}

\begin{frame}
\frametitle{Ook hebben we leren citeren.}
"Ik heb geen idee wat deze persoon heeft gezegd maar ik citeer hem wel in apa style"\cite{braams:babel}

\end{frame}

\begin{frame}
\frametitle{References}
\bibliography{mybib}
\bibliographystyle{apacite}
\end{frame}

\end{document}